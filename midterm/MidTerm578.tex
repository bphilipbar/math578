%% AMS-LaTeX Created with the Wolfram Language for Students - Personal Use Only : www.wolfram.com

\documentclass{article}
\usepackage{amsmath, amssymb, graphics, setspace}

\newcommand{\mathsym}[1]{{}}
\newcommand{\unicode}[1]{{}}

\newcounter{mathematicapage}
\begin{document}

\section*{Numerical Solutions of Shallow Water Equations}

Math 578 * University of New Mexico * October 2016

Students: Oleksii Beznosov $\&$ Brad Philipbar

\pmb{ Abstract:} \\
Systems of Shallow Water Equations (SSWE) model the propagation of disturbances in water and other incompressible fluids and are used to describe
the dynamics of important phenomenon. Models of the 2D type constitute one of the widest classes of models studied in engineering. The underlying
assumption is that the depth of the fluid is small compared to the wave length of the disturbance. Initially a look at conservation in the absence
of the Coriolis force will be describe. Specifying either periodic boundary conditions, or free boundary conditions for h, and reflective boundary
conditions for uh and uv. 

\\
\pmb{ Introduction:} \\
Systems of Shallow Water Equations (SSWE) model the propagation of disturbances in water and other incompressible fluids and are used to describe
the dynamics of important phenomenon like tsunami.The underlying assumption is that the depth of the fluid is small compared to the wave length of
the disturbance. The conservative form of the shallow water equations is,

\(\begin{array}{ll}
 \{ & 
\begin{array}{ll}
 \frac{\partial h}{\partial t}+\frac{\partial (\text{hu})}{\partial x}+\frac{\partial (\text{hv})}{\partial y}=0\frac{\partial (\text{hu})}{\partial
t}+\frac{\partial \left(\text{hu}^2+\frac{1}{2}\text{gh}^2\right)}{\partial x}+\frac{\partial (\text{huv})}{\partial y}=\text{fhv}\frac{\partial
(\text{hv})}{\partial t}+\frac{\partial (\text{huv})}{\partial x}+\frac{\partial \left(\text{hv}^2+\frac{1}{2}\text{gh}^2\right)}{\partial y}=-\text{fhu}
& \text{Governing} \text{Equations}(1) \\
\end{array}
 \\
\end{array}\)

Here h$>$0 is the fluid height, u and v are the horizontal and vertical velocities, g is the acceleration due to gravity (9.8m/s${}^{\wedge}$2 on
Earth) and f is the Coriolis force. We let

\(U=\left(
\begin{array}{c}
 h \\
 \text{hu}\text{hv} \\
\end{array}
\right), F(U)=\left(
\begin{array}{c}
 \text{hu} \\
 \text{hu}^2+\frac{1}{2}\text{gh}^2 \\
 \text{huv} \\
\end{array}
\right), G(U)=\left(
\begin{array}{c}
 \text{hv} \\
 \text{huv} \\
 \text{hv}^2+\frac{1}{2}\text{hg}^2 \\
\end{array}
\right), S(u)=\left(
\begin{array}{c}
 0 \\
 \text{fhv} \\
 -\text{fhu} \\
\end{array}
\right)\), { } { }(2)

to rewrite the equations in compact form,

\(\frac{\partial U}{\partial t}+\frac{\partial F(U)}{\partial x}+\frac{\partial G(U)}{\partial y}=S(U).\)$\quad \quad \quad \quad $(3) { }

In the absense of the Coriolis force, we get the standard form of the conservtion law.

\(\frac{\partial U}{\partial t}+\frac{\partial F(U)}{\partial x}+\frac{\partial G(U)}{\partial y}=0.\)$\quad \quad \quad \quad $(4)

We specify either periodic boundary condtions, or {``}free{''} boundary conditions for h and {``}reflective{''} boundary for uh and uv. Free boundary
conditions means the boundary exerts no stress while reflective boundary conditions means the boundary behaves like a mirror. { }

Discretization:\\
We solve the shallow water equations using the Lax-Wendroff method and Richmeyer{'}s two step method. The Lax-Wendroff shallow water equations in
1D are, 

\(\frac{\partial U}{\partial t}+\frac{\partial F(U)}{\partial x}=0.\) { } { } (5)

where \(U=[h,\text{hu}]^T\) and \(F(U)=\left[\text{hu}, \text{hu}^2+\frac{1}{2}\text{gh}^2\right]^T\). The Lax-Wendroff method for the system (5)
is

\(U_i^{n+1}=U_i^n-\frac{\text{$\Delta $t}}{2}\left(\left(I-\frac{\text{$\Delta $t}}{\text{$\Delta $x}}A_{i+1/2}^n\right)\left(D_+F_i^n\right)+\left(\left(I+\frac{\text{$\Delta
$t}}{\text{$\Delta $x}}A_{i-1/2}^n\right)\left(D_{\_}F_i^n\right)\right)\right.\) { } { }(6)

where \(F_i^n=F\left(U_i^n\right), A_{i+1/2}\) is the Jacobian matrix of F evaluated at \(U_{i+1/2}\) and \(D_+\) and \(D_-\) are the standard forward
and backward difference operators defined as,

\(D_{\pm }w(x)=\frac{\pm w(x\pm \text{$\Delta $x})-w(\unicode{f39e}\pm w(x))}{\text{$\Delta $x}}\). { } { } { } { } (7)

Assuming we have \(N_x\) points in the x-direction we with \(\Omega =[0\text{  }1]\), so that \(1\leq i\leq n\). We can impose periodic boundary
conditions by,

\(U_1=U_{N_x}\).

Reflective boundary condtions at x=0 are imposed by,

\(U_1=-U_2\).

Free boundary conditions at \(x=0\) are implemented as,

\(U_1=U_2\).

It should be noted that the above equations describe how these conditions are imposed on the entire vector U, here we implement reflective boundary
conditions for hu and free boundary conditions for (h). 

\\


\pmb{ Lax-Wendroff, Derivation and Implementation for 1D Solver:}

\pmb{ Von-Neumann stability analysis:}

\(u=e^{\text{at}}e^{\text{ikx}}\) for \(\frac{\partial a}{\partial t}+\frac{\partial u}{\partial x}=0\)\\
\(e^{\text{a$\Delta $t}}=\left[1-2r^2\sin \left(\frac{\text{$\phi \Delta $x}}{2}\right)\sin (\text{k$\Delta $x})\right]-\text{irsin}(\text{k$\Delta
$x})<1\)\\
$\Rightarrow $r$>>$1 ?

The Lax-Wendroff system of equation(5) is hyperbolic, and has a Jacobian \(A(U)\) of \(F(U)\) that has real eigenvalues and a full set of eigenvectors.
We must first re-write the equations of system (5) so that it is hyperbolic,

\(F(U)=\left(
\begin{array}{c}
 \text{hu} \\
 \text{hu}^2+\frac{1}{2}\text{gh}^2 \\
\end{array}
\right)=\left(
\begin{array}{c}
 u_2 \\
 u_1^2+\frac{1}{2}\text{gu}_1^2 \\
\end{array}
\right)\), note \(u=\left(
\begin{array}{c}
 u_1 \\
 u_2 \\
\end{array}
\right)=\left(
\begin{array}{c}
 h \\
 \text{hu} \\
\end{array}
\right)\). 

\(A(U)=\left(
\begin{array}{cc}
 \frac{\partial F_1}{\partial u_1} & \frac{\partial F_1}{\partial u_2} \\
 \frac{\partial F_2}{\partial u_1} & \frac{\partial F_2}{\partial u_2} \\
\end{array}
\right)=\left(
\begin{array}{cc}
 0 & 1 \\
 -u+\text{gh} & 2u \\
\end{array}
\right)\Rightarrow \lambda =u\pm \sqrt{\text{gh}}.\)

We can see we now have real eigenvalues (\(\lambda =u\pm \sqrt{\text{gh}}\)), and a full set of eigenvectors. This new modified form becomes our
new F.

\textit{ Is above enough to explain how we compute the Jacobian matrix?}

Choosing the interesting initial conditions, \(h=4+\sin (2\text{$\pi $x}) \text{for}\text{  }u=0\) and \(h=e^{-\left(\frac{x-\mu }{\sigma }\right)^2}
\text{for}\text{  }u=0\); a combination of the boundary condtions periodic for both h and hu, and free for h, reflective for uh are computed using
a solver written in Python.

\pmb{ Conservation Laws 1D:}

We expect the following quantities to be conserved: mass h, momentum or mass velocity, hu and hv, energy \(0.5\left(\text{hv}^2+\text{gh}^2\right)\).
Potential vorticity is only looked at in the 2D solver. 

Invesigation of conservation of momentum in solver. Is the momentum conserved, what explanation do we have?



\pmb{ 2D Solver:}

The spatial domain for the 2D solver is \(\Omega =[0\text{  }1]\times [0\text{  }1].\) To go about solving the system in 2D there are many options,
two are presented here. Using the 1D equations of the Lax-Wendroff scheme for system (4) a dimenional split can be implemented by doing a step in
x, followed by a step in y: 

\(U_{i,j}^*=U_{i,j}^n-\frac{\text{$\Delta $t}}{2}\left(\left(I-\frac{\text{$\Delta $t}}{\text{$\Delta $x}}A_{i+1/2,j}^n\right)D_+^xF_{i,j}^n+\left(I+\frac{\text{$\Delta
$t}}{\text{$\Delta $x}}A_{i-1/2,j}^n\right)D_{\_}^xF_{i,j}^n\right)\),\\
\(U_{i,j}^{n+1}=U_{i,j}^*-\frac{\text{$\Delta $t}}{2}B\left(I-\frac{\text{$\Delta $t}}{\text{$\Delta $y}}B_{i,j+1/2}^*\right)D_+^yG_{i,j}^*+\)\(\left.\left(I+\frac{\text{$\Delta
$t}}{\text{$\Delta $x}}B_{i,j-1/2}^*\right)D_{\_}^yF_{i,j}^*\right)\).

The boundary conditions are prescribed for two cases. The first, periodic for both h, hu, and hv, and 2. The second, and more complicated conditions,
free for (h), reflective in the horizontal direction and free in the vertical direction for (uh), and reflective in the vertical direction and free
in the horizontal direction for (vh).\\
The initial conditions are chosen to be piecewise, 

\(u_0(x,y)=
\begin{array}{ll}
 \{ & 
\begin{array}{ll}
 8, & \text{if} (x-0.3)^2+(y-0.3)^2 \\
 1, & \text{otherwise} \\
\end{array}
 \\
\end{array}\)

interestingly forming a cylindrical column. 

\end{document}
